% !TEX program    = pdflatex
% !TEX encoding   = UTF-8
% !TEX spellcheck = de_DE

\documentclass[11pt,twosided=true]{mathphys-protocol}
%-------------------------------------------------
% Informationen zur Sitzung, Werte anpassen!
%-------------------------------------------------
\renewcommand{\simo}{Wunderbare Simo}
\renewcommand{\protokoll}{Ehrenprotokollant*in}
\renewcommand{\sbegin}{18:15 Uhr}
\renewcommand{\sende}{xx:xx Uhr}
\renewcommand{\datum}{tt.mm.yyyy}
% {Namen der anwesenden FSR}{Sind die FSR beschlussfähig? ("Beschlussfähig" wenn 2 von 3 da, sonst "Nicht Beschlussfähig"}
\FSRPhysik{}{(Nicht) Beschlussfähig}
\FSRMathe{}{(Nicht) Besschlussfähig}
\FSRInfo{}{(Nicht) Beschlussfähig}

% Konsenstufen 1-6 in geschweifte Klammern eintragen 

\begin{document}

\maketitlepage

\Tagesordnung{1} % Konsensstufe 1-6

\nextsimo{Name} % SiMo nachste Woche einfugen

\section{Protokolle}

\protokollbeschluss{Datum}{1} % Datum und Konsensstufe
% Der Beschluss des Protokolls vom tt.mm.yyyy wird auf die nächste Sitzung vertagt.

%%%% Achte ab hier darauf, dass Dopplungen der Standard TOPs gelöscht werden. %%%%

\section{Ein antrag}
\begin{antrag}
Die Fachschaft macht was auch immer und ist cool.
\end{antrag}
%Abstimmungsergebnis
\konsens{3}

\section{Finanzantrag}
\Antrag{% Summe, Hausholtsposten, Zweck
    \finanzantragstext{666€}{xxx.0217}{Anschaffung eines Blåhajs}%
}{%
    Hier eine GUTE Begründung.%
}{%
    \FSReinstimmig\\% wichtig!!!!!
    \konsens{2}%
}

\section{Aktuelles aus Studium und Lehre}

\section{Berichte}


\section{Sonstiges}

\emph{Die Sitzungmoderation schließt die Sitzung um \sende.}
\end{document}
