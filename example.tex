% !TEX program    = pdflatex
% !TEX encoding   = UTF-8
% !TEX spellcheck = de_DE

\documentclass[11pt, fachschaft=mathphys,twosided=true]{mathphys-protocol}
\usepackage{booktabs}
\usepackage{hyperref}
\renewcommand\thesection{TOP \arabic{section}:}
\renewcommand*\thesubsection{TOP \arabic{section}.\arabic{subsection}:}
\renewcommand\contentsname{Tagesordnung}
\newenvironment{antrag}{\begin{quote}\begin{itshape}}{\end{itshape}\end{quote}}

%-------------------------------------------------
% Konsensvorlagen (ggf. anpassen!)
%-------------------------------------------------
\newcommand{\konsens}[1]{In der Fachschaftssitzung MathPhysInfo, sowie in den anwesenden Fachschaftsräten, besteht Konsens ohne Bedenken.\\} % immer die Anzahl der Anwesenden anpassen!
\newcommand{\konsensLB}[1]{In der Fachschaftssitzung MathPhysInfo, sowie in den anwesenden Fachschaftsräten, besteht Konsens mit leichten Bedenken.\\} % immer die Anzahl der Anwesenden anpassen!
\newcommand{\konsensE}[1]{In der Fachschaftssitzung MathPhysInfo, sowie in den anwesenden Fachschaftsräten, besteht Konsens mit Enthaltung.\\} % immer die Anzahl der Anwesenden anpassen!
\newcommand{\konsensFsrPhys}{Die Fachschaftsratssitzung Physik entscheidet einstimmig, den Beschluss entsprechend der Entscheidung der Fachschaftssitzung MathPhysInfo umzusetzen.\\}
% \newcommand{\konsensFsrMathe}{Die Fachschaftsratssitzung Mathematik entscheidet einstimmig, den Beschluss entsprechend der Entscheidung der Fachschaftssitzung MathPhysInfo umzusetzen.\\}
\newcommand{\konsensFsrInfo}{Die Fachschaftsratssitzung Informatik entscheidet einstimmig, den Beschluss entsprechend der Entscheidung der Fachschaftssitzung MathPhysInfo umzusetzen.\\}

\setlength{\parindent}{0pt}
\setlength{\parskip}{1em}


%-------------------------------------------------
% Informationen zur Sitzung, Werte anpassen!
%-------------------------------------------------
\newcommand{\simo}{Kai-Uwe}
\newcommand{\tipper}{Max Müller}
\newcommand{\sbegin}{18:15 Uhr}
\newcommand{\sende}{xx:xx Uhr}
\def\datum{08.02.2023}
\def\FSRPhysik{FSR 1, FSR 2}
\def\FSRMathe{FSR 1, FSR 2}
\def\FSRInfo{FSR 1, FSR 2}
\def\besschlussf{JA}

\begin{document}
\date{\vspace{-2em} \datum \vspace{-1em}} % Datum ersetzen
\title{\vspace{-2em}Vorläufiges Protokoll der Fachschaftssitzung MathPhysInfo}
\maketitle

\begin{tabbing}
    \textbf{Sitzungsmoderation:}\quad\=\simo \\% SiMo einfügen
    \textbf{Protokoll:}\> \tipper \\% Protokoll einfügen
    \textbf{Beginn:}\>\sbegin\\
    \textbf{Ende:}\>\sende\\ % Sitzungsende einfügen
\end{tabbing}

\section{Begrüßung}
    Die Sitzungsmoderation begrüßt die anwesenden Mitglieder der Studienfachschaften Mathematik, Physik und Informatik und eröffnet so die Fachschaftsvollversammlung der Studienfachschaften Mathematik, Physik und Informatik.

\section{Feststellung der Beschlussfähigkeiten}
    Fachschaftsrat Physik, Mathe und Informatik sind alle Beschlussfähig.

\section{Beschluss des Protokolls der letzten Sitzung}

\begin{antrag}
    Annahme des Protokolls vom xx. Monat 2019. \\% Datum einfügen
\end{antrag}
\konsensE{}

\section{Feststellen der Tagesordnung}
\begin{antrag}
    Die Tagesordnung wird in der vorliegenden Form angenommen.
\end{antrag}
\konsens{}

\section{Sitzungsmoderation für die nächste Sitzung}
    Die Sitzungsmoderation für die Fachschaftssitzung MathPhysInfo der nächsten Woche wird von xxx übernommen. % SiMo nachste Woche einfugen

%%%% Achte ab hier darauf, dass Dopplungen der Standard TOPs gelöcht werden. %%%%
%$sections
\section{Finanzanträge}
\textbf{Beschluss:} Um was geht es bei dem Finanzantrag?\\
\textbf{Datum:} \datum \\
\textbf{Postennr. im Haushaltsplan:} xxx.xxxx\\
\textbf{Betrag:} 0 €\\
\textbf{Ergebnis der Abstimmung:} EINSTIMMIG oder JA/NEIN/ENTHALTUNG\\
\subsection*{Beschlusstext:}

\subsection*{Begründung:} TEXT


\emph{Die Sitzungmoderation schließt die Sitzung um \sende.}

\end{document}
