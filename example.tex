% !TEX program    = pdflatex
% !TEX encoding   = UTF-8
% !TEX spellcheck = de_DE

\documentclass[11pt, twosided=true]{mathphys-protocol}


%-------------------------------------------------
% Informationen zur Sitzung, Werte anpassen!
%-------------------------------------------------
\renewcommand{\simo}{Miriam Kaden}
\renewcommand{\protokoll}{Farah Alam}
\renewcommand{\sbegin}{18:15 Uhr}
\renewcommand{\sende}{19:33 Uhr}
\renewcommand{\datum}{18.01.2023}
% {Namen der FSR}{Sind die FSR beschlussfähig? ("Beschlussfähig" wenn 2 von 3 da, sonst "Nicht Beschlussfähig"}
\FSRPhysik{Kim Schramer, Lea Bartels}{Beschlussfähig}
\FSRMathe{Farah Alam, Arianit Miftari}{Beschlussfähig}
\FSRInfo{Nikolai Smolkin, Max Wipplinger}{Beschlussfähig}

\begin{document}

\maketitlepage

\Tagesordnung{1} % Konsensstufe 1-6

\nextsimo{Arianit Miftari} % SiMo nachste Woche einfugen

\section{Protokolle}
    Der Beschluss des Protokolls der Fachschaftssitzung vom 11.01.2023 wird vertagt.

%%%% Achte ab hier darauf, dass Dopplungen der Standard TOPs gelöscht werden. %%%%

\section{MathPhysTheo Impressionen}
Die MathPhysTheo war als Veranstaltung für unsere Besucher ein voller Erfolg. Es wird sich bei allen Beteiligten bedankt. Eine Nachbesprechung sowie die finanzielle Auswertung finden die nächsten Wochen statt.

\section{Arbeiter*innenkindreferat }
Die Fachschaft MathPhysInfo spricht sich dafür aus, als politisch neutrale Instanz im StuRa einen politisch-neutralen formulierten und begründeten Antrag auf ein Arbeiterkind-Referat einzubringen und damit sich für die Belange sozioökonomisch benachteiligter Studierender einzusetzen.

\section{Aktuelles aus Studium und Lehre}
Es wieder ein Sommerfest stattfinden. Hierfür werden Helfer gesucht.

\section{Berichte}
    \begin{itemize}
        \item Die QSM wurden beantragt. In der Mathematik und Informatik gab es Ungenauigkeiten in der Mittelzuweisung für das Wintersemester, weshalb diese Anträge erst einmal nicht eingereicht wurden. Es bleibt nun bis zum 15.05. Zeit hier nachzubessern.
        \item Im StuRa gibt es einen Antrag betreffend der Nachhaltigkeit im Einkauf vom Fleisch. Hier kann gerne Input an die Vertreter im StuRa weitergereicht werden.
    \end{itemize}

\section{Sonstiges}
Felix legt das im Root-Team nieder. Vielen Dank Felix für deine Arbeit.
Ein Nachfolger, der das Team künftig unterstützt, wird gesucht.

\emph{Die Sitzungmoderation schließt die Sitzung um \sende.}
\end{document}


